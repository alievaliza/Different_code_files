\documentclass[12pt]{article}
\usepackage[T2A]{fontenc}
\usepackage[utf8]{inputenc}
\usepackage[russian]{babel}
\usepackage[a4paper,left=2.5cm, right=1.5cm, top=2.5cm, bottom=2.5cm]{geometry}
\usepackage{graphicx}

\begin{document}
\title{Пример документа LaTeX}
\author{А.В. Тор}
\date{\today}
\maketitle

\abstract{Здесь помещается текст аннотации.}

\section{Текст}

Абзацы при наборе разделяются пустой строкой. По умолчанию, TeX выровняет текст по ширине. 

Шрифт бывает \textbf{полужирным}, \textit{наклонным}, \texttt{моноширинным} и ещё много каким. Размеры его изменяются от {\tiny малюсенького} до {\Huge огромного}.

Мы познакомились с тем, как:

\begin{itemize}
\item набирать текст;
\item управлять начертанием и размером шрифта;
\item а теперь ещё и создали список.
\end{itemize}

\section{Немного математики}

Формулу можно поместить внутрь текста: \( y=\sin x \) или сделать отдельным абзацем:

\[
\int_a^b x^2 dx = \left. \frac{x^3}{3} \right|_a^b .
\]

\section{Иллюстрации}

Вставим рисунок и сошлёмся на него (рис.~\ref{lion})

\begin{figure}[h]
\center{\includegraphics[scale=0.5]{TeX_lion.jpg}}
\caption{Талисман \TeX, созданный художником Дуэйном Бибби}
\label{lion}
\end{figure}

\end{document}